\documentclass[a4paper,10pt]{article}

\usepackage[T1]{fontenc}
\usepackage[utf8x]{inputenc}
\usepackage[ngerman]{babel}
\usepackage{textcomp}


\begin{document}

\section{Beitragsordnung}

\subsubsection*{§ 1 Beitragssätze}
Die Beiträge sind wie folgt gestaffelt:
\begin{enumerate}
  \item Privatperson: 30\textsf{\texteuro}
  \item Privatperson (ermäßigt)\footnote{Berechtigte sind Kinder und Jugendliche
    bis zur Vollendung des 18. Lebensjahres, Personen in Ausbildung, sowie
  Empfänger von staatlichen Sozialleistungen.}: 15\textsf{\texteuro}
\item Familie\footnote{Unter den Familienbeitrag fallen Familien mit Kinder bis
    zur Vollendung des 18. Lebensjahres bzw. in Ausbildung bis zur Vollendung des
    27. Lebensjahr. Jedes Familienmitglied ist separates Mitglied mit
  Stimmrecht.}: 40\textsf{\texteuro}
\item Gemeinnützige Vereine: 50\textsf{\texteuro}
\item Firmen und andere juristische Personen: 600\textsf{\texteuro}
\end{enumerate}
Auf eigenen Wunsch kann jedes Mitglied bei Anmeldung oder nachträglich den zu
entrichtenden Beitrag freiwillig erhöhen.


\subsubsection*{§ 2 Nachweise}
Auf Verlangen des Vorstandes müssen die Mitglieder Nachweise für die
Zugehörigkeit zur gewählten Beitragsklasse vorlegen.


\subsubsection*{§ 3 Ermäßigung}
\begin{enumerate}
  \item Der Vorstand kann, insbesondere zum Zweck der Mitgliedergewinnung,
Beitragsermäßigungen genehmigen. Die Beitragsermäßigungen gelten jeweils für ein
Kalenderjahr.
\item Für Mitglieder, die nach dem 30. Juni in den Verein eintreten, wird im
  laufenden Kalenderjahr nur die Hälfte des Mitgliedsbeitrags erhoben.
\end{enumerate}


\subsubsection*{§ 4 Zahlung}
\begin{enumerate}
  \item Die Mitgliedsbeiträge werden kalenderjährlich, d.h. vom 1.1. bis 31.12. erhoben.
  \item Monatsbeiträge sind nicht vorgesehen.
  \item Die Beiträge sind per Bankeinzugsverfahren oder per Überweisung zu entrichten.
Eine Überweisung ist zeitnah zu tätigen.
\item Die Mitgliedschaft wird automatisch verlängert.
\item Kosten für eine geplatze Lastschrift bzw. Lastschriftrückbuchung trägt das
Mitglied.
\end{enumerate}

\subsubsection*{§ 5 Fälligkeit}
Die Mitgliedsbeiträge sind bei Bestandsmitgliedern zum 1.1. des Jahres und bei
Neumitgliedern zum Datum der Aufnahme in den Verein fällig.

\subsubsection*{§ 6 Kündigung}
Die Regeln zur Kündigung finden sich in der Satzung.

\subsubsection*{§ 7 Inkrafttreten}
Die Beitragsordnung tritt mit der Vereinsgründung in Kraft.

\end{document}
