\documentclass[a4paper,10pt]{article}

\usepackage[T1]{fontenc}
\usepackage[utf8x]{inputenc}
\usepackage[ngerman]{babel}

\begin{document}

\section{Satzung}

\subsubsection*{§ 1 Name, Sitz und Geschäftsjahr}
\begin{enumerate}
   \item Der Verein führt den Namen ``Computerverein Passau'' (compass). Er soll
     in das Vereinsregister eingetragen werden. Nach der Eintragung führt er
     zusätzlich das Kürzel ``e.V.'' im Namen.
   \item Der Sitz des Vereins ist Passau.
   \item Das Geschäftsjahr des Vereins ist das Kalenderjahr.
\end{enumerate}

\subsubsection*{§ 2 Vereinszweck }
\begin{enumerate}
  \item Zweck des Vereins ist die Förderung der Erziehung sowie der Volks- und Berufsbildung.
  \item Die Vereinsziele werden insbesondere erreicht durch:
    \begin{itemize}
      \item Wissensvermittlung in den Bereichen digitale Technologien und neue Medien.
      \item Durchführung von Bildungsveranstaltungen und Workshops zu den oben
        genannten Themenbereichen für interessierte Bevölkerungskreise,
        beispielsweise Kinder und Jugendliche.
      \item Kooperationen mit Schulen, Bildungs- und Forschungseinrichtungen.
      \item Förderung eines mündigen, verantwortungsvollen, aufgeklärten und
        kreativen Umgangs mit Technologie.
      \item Bildung von speziellen Interessens- und Projektgruppen.
    \end{itemize}
  \item Der Verein ist politisch und konfessionell neutral.
\end{enumerate}

\subsubsection*{§ 3 Gemeinnützigkeit}
Der Verein verfolgt ausschließlich und unmittelbar gemeinnützige Zwecke im Sinne
des Abschnitts ``Steuerbegünstigte Zwecke'' der Abgabenordnung. Der Verein ist
selbstlos tätig; er verfolgt nicht in erster Linie eigenwirtschaftliche Zwecke.
Mittel des Vereins dürfen nur für die satzungsgemässen Zwecke verwendet werden.
Die Mitglieder erhalten keine Zuwendungen aus Mitteln des Vereins. Es darf keine
Person durch Ausgaben, die dem Zweck der Körperschaft fremd sind, oder durch
unverhältnismässig hohe Vergütungen begünstigt werden.

\newpage

\subsubsection*{§ 4 Mitglieder}
\begin{enumerate}
  \item Mitglied können natürliche und juristische Personen werden.
  \item Der Antrag auf Aufnahme in den Verein ist schriftlich oder elektronisch beim
Vorstand einzureichen. Über die Aufnahme entscheidet der Vorstand.
\item Die Mitgliedschaft endet mit dem Tod, durch Austritt, durch Streichung oder
durch Ausschluss aus dem Verein. Der Austritt ist unter Einhaltung einer
Kündigungsfrist von einem Monat nur zum Schluss eines Kalenderjahres möglich.
\item Der Austritt ist schriftlich gegenüber dem Vorstand zu erklären.
\item Über den Ausschluss entscheidet der Vorstand.
\item Gegen die Ablehnung der Aufnahme und gegen den Ausschluss kann Berufung zur
nächsten Mitgliederversammlung eingelegt werden.
\item Ein Mitglied kann seine Mitgliedschaft durch einen Beschluss des Vorstandes
verlieren, nachdem der Vorstand festgestellt hat, dass das Mitglied mehr als ein
halbes Jahr beitragssäumig ist und/oder wenn die Wohnanschrift des Mitglieds
nicht mehr zu ermitteln ist (Streichung).
\item Vor der Beschlussfassung ist dem Mitglied von der geplanten Streichung durch
eingeschriebenen Brief an die letzte bekannte Adresse Gelegenheit zur
Stellungnahme zu geben.
\item Ausgeschiedene Mitglieder haben aus ihrer Mitgliedschaft keinen Anspruch auf das
Vereinsvermögen.
\end{enumerate}



\subsubsection*{§ 5 Mitgliedsbeiträge}
\begin{enumerate}
  \item Von den Mitgliedern wird ein Mitgliedsbeitrag erhoben. Die Höhe des Beitrags
richtet sich nach der aktuell gültigen Beitragsordnung.
\item Zur Finanzierung besonderer Vorhaben oder zur Beseitigung spezieller
Schwierigkeiten des Vereins können auch Umlagen erhoben werden.
\item Die Beitragsordnung und die Höhe der Umlagen wird in der
Mitgliederversammlung durch eine 2/3 Mehrheit beschlossen.
\item Die Höhe der Umlage darf die Höhe der individuellen Mitgliedsbeiträge und den
Betrag von 50.- Euro nicht übersteigen.
\item Auf Erstattung bereits geleisteter Mitgliedsbeiträge besteht nach
  Ausscheiden eines Mitglieds kein Anspruch.
\end{enumerate}


\subsubsection*{§ 6 Organe}
Organe des Vereins sind die Mitgliederversammlung, der Beirat und der Vorstand.

\newpage

\subsubsection*{§ 7 Vorstand}
\begin{enumerate}
  \item Der Vorstand besteht aus dem Vorsitzenden, dem stellvertretenden Vorsitzenden,
dem Schriftführer und dem Kassierer sowie bis zu drei Beisitzern.
\item Vorstand im Sinne des § 26 BGB sind der Vorsitzende, der stellvertretende
Vorsitzende und der Kassierer. Jeder ist allein vertretungsberechtigt.
\item Die Amtszeit des Vorstands beträgt zwei Jahre. Sie beginnt mit dem Tag der Wahl.
Der Vorstand bleibt auch nach Ablauf der Amtszeit bis zur Neuwahl im Amt.
\item Der Vorstand kann auch vor Ablauf der Amtszeit durch die Mitgliederversammlung
abberufen werden. Eine Abberufung betrifft stets den gesamten Vorstand. In
diesem Fall endet die Amtszeit unverzüglich.
\item In den Vorstand können nur volljährige Vereinsmitglieder gewählt werden.
\end{enumerate}


\subsubsection*{§ 8 Geschäftsführung des Vorstands}
\begin{enumerate}
  \item Die Sitzungen des Vorstands werden vom Vorsitzenden, bei seiner Verhinderung vom
stellvertretenden Vorsitzenden, mindestens eine Woche vor dem Termin der Sitzung
unter Angabe von Zeit und Ort einberufen.
\item Die Vorstandssitzung ist beschlussfähig, wenn mindestens 2/3 der
Vorstandsmitglieder anwesend sind.
\item Der Vorstand gibt sich eine Geschäftsordnung, die die weitere Geschäftsführung
regelt. Sie kann zusätzlich höhere Anforderungen als in Abs. 1 und 2 angegeben
stellen. Die Geschäftsordnung ist den Mitgliedern elektronisch bekanntzugeben.
\item Kommt es bei der Abstimmung über die Geschäftsordnung zu Stimmengleichheit, so
entscheidet die Stimme des Vorsitzenden.
\end{enumerate}



\subsubsection*{§ 9 Kassenführung und Kassenprüfung}
\begin{enumerate}
  \item Die zur Erreichung des Vereinszwecks erforderlichen Mittel werden in erster
Linie aus Mitgliedsbeiträgen, Spenden und weiteren Einnahmen aufgebracht.
\item Der Kassierer hat über die Kassengeschäfte Buch zu führen und eine
Jahresrechnung zu erstellen.
\item Zahlungen dürfen nur auf Grund von Auszahlungsanordnungen des Vorsitzenden oder
- bei dessen Verhinderung - des stellvertretenden Vorsitzenden geleistet werden.
\item Die Jahresrechnung ist von zwei Kassenprüfern, die jeweils auf 2 Jahre
  gewählt werden, zu prüfen. Die Kassenprüfer dürfen dem Vorstand nicht
  angehören. Die Kassenprüfung ist der Mitgliederversammlung zur Genehmigung
  vorzulegen.
\end{enumerate}


\subsubsection*{§ 10 Mitgliederversammlung}
\begin{enumerate}
  \item Zuständigkeit der Mitgliederversammlung:
    \begin{itemize}
      \item Wahl und Abberufung der Vorstandsmitglieder.
      \item Wahl der Kassenprüfer.
      \item Entgegennahme der Berichte des Vorstands.
      \item Festsetzung von Beiträgen und Umlagen.
      \item Entlastung des Vorstands.
      \item Beschlussfassung über Änderungen der Satzung und über die Auflösung des Vereins.
      \item Beschlussfassung über die Berufung gegen einen Beschluss des Vorstands über
einen abgelehnten Aufnahmeantrag und über einen Ausschluss.
\end{itemize}
\item Die Mitgliederversammlung findet einmal jährlich statt. Die
Mitgliederversammlung muss außerdem einberufen werden, wenn das Interesse des
Vereins es erfordert oder 1/10 der Mitglieder es schriftlich unter Angabe der
Gründe vom Vorstand die Einberufung fordert.
\item Die Mitgliederversammlung wird vom Vorsitzenden, bei seiner Verhinderung vom
stellvertretenden Vorsitzenden unter Angabe von Zeit, Versammlungsort und
Tagesordnung mindestens zwei Wochen vor dem Versammlungstermin einberufen.
\item Die Einladung erfolgt als E-Mail, auf besonderen Wunsch des Mitglieds auch per
Post oder Fax. Das Mitglied ist verpflichtet, dem Vorstand eine E-Mail-Adresse
bzw. Postadresse oder Faxnummer mitzuteilen, unter der es erreichbar ist. Die
Einladung gilt bei Versand an die gemeldete E-Mail-Adresse, Postadresse oder
Faxnummer als zugestellt.
\item Jedes Mitglied kann bis spätestens eine Woche vor dem Tag der
Mitgliederversammlung beim Vorsitzenden schriftlich beantragen, dass weitere
Angelegenheiten nachträglich auf die Tagesordnung gesetzt werden. Über Anträge
auf Ergänzung der Tagesordnung, die erst in der Versammlung gestellt werden,
beschliesst die Mitgliederversammlung.
\item Über die Versammlung ist eine Niederschrift anzufertigen, die mindestens Zeit
und Ort der Versammlung, den Namen des Versammlungsleiters, alle gefassten
Beschlüsse, die Abstimmungsergebnisse und ein Verzeichnis der anwesenden
Mitglieder enthält.
\item Die Versammlung wird vom Vorsitzenden, bei seiner Verhinderung vom
stellvertretenden Vorsitzenden geleitet. Bei Wahlen soll die Versammlungsleitung
für die Dauer der Wahl einer anderen Person übertragen werden, die selbst nicht
zur Wahl steht.
\end{enumerate}

\newpage

\subsubsection*{§ 11 Beschlussfassung der Mitgliederversammlung}
\begin{enumerate}
  \item Jede ordnungsgemäß einberufene Mitgliederversammlung ist unabhängig von der
Anzahl der erschienenen Mitglieder beschlussfähig.
\item Die Versammlung wird beschlussunfähig, sobald mindestens die Hälfte der
erschienenen Mitglieder die Versammlung verlassen hat.
\item Die Mitgliederversammlung fasst ihre Beschlüsse mit Ausnahme von
  Abberufung des Vorstands, Änderung der Satzung, Beitragsordnung und
  Umlagen sowie Auflösung des Vereins mit einfacher Mehrheit.  Bei
  Stimmengleichheit gilt der Antrag als abgelehnt.
\item Die Beschlussfassung zur Abberufung des Vorstands, Satzungsänderung erfolgt mit
einer Mehrheit von mindestens 2/3 der Stimmen. Eine Beschlussfassung zur
Satzungsänderung ist nur zulässig, falls der Wortlaut des Antrags zur
Satzungsänderung in der Einladung aufgeführt war.
\item Die Beschlussfassung erfolgt durch Handzeichen, auf Antrag mindestens eines
Mitglieds in geheimer Abstimmung. Vorstandswahlen und Abstimmungen zur
Abberufung des Vorstands sind stets in geheimer Abstimmung durchzuführen.
\item Stimmberechtigt sind alle Mitglieder ab Vollendung des 14. Lebensjahres.
\item Stimmübertragungen sind nicht zulässig.
\end{enumerate}


\subsubsection*{§ 12 Beirat}
\begin{enumerate}
  \item Der Vorstand kann zu seiner Unterstützung bei der Wahrnehmung seiner Aufgaben
und zu seiner Beratung in einer Angelegenheit des Vereins einen Beirat bilden.
\item Die Beiräte müssen nicht Vereinsmitglieder sein.
\item Die Bestellung erfolgt durch Beschluss des Vorstandes.
\item An den Sitzungen des Beirates nimmt ein Vorstandsmitglied mit Sitz und Stimme
teil. Er bereitet die Sitzungen des Beirates vor und leitet die Empfehlungen des
Beirates an den Vorstand weiter.
\end{enumerate}


\subsubsection*{§ 13 Jugendgruppe}
Der Verein führt eine eigenständige Jugendgruppe. Näheres regelt die
Jugendordnung.

\newpage

\subsubsection*{§ 14 Auflösung des Vereins}
\begin{enumerate}
  \item Die Auflösung des Vereins kann nur in einer zu diesem Zweck einberufenen
Mitgliederversammlung mit einer Mehrheit von mindestens 3/4 der Stimmen
beschlossen werden.
\item Bei Auflösung des Vereins oder bei Wegfall steuerbegünstigter Zwecke fällt das
Vermögen des Vereins an eine gemeinnützige Körperschaft, die von der
Mitgliederversammlung bestimmt wird. Das Vermögen ist unmittelbar und
ausschliesslich für gemeinnützige Zwecke im Sinne dieser Satzung zu verwenden.
\end{enumerate}

\end{document}
